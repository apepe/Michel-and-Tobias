We now utilize the same method to calculate expected visibilities of both dipole and quadrupole modes in stars with $1.25 \, M_\odot \leq M \leq 3 \, M_\odot$. The ratio of suppressed mode power to normal mode power is
\begin{equation}
\label{eqn:vsup}
\frac{V_{\rm sup}^2}{V_{\rm norm}^2} = \bigg[ 1 + \Delta \nu \,\tau \,T^2 \bigg]^{-1} \, .
\end{equation}
Here, $\Delta \nu$ is the large frequency separation, $\tau$ is the radial mode lifetime, and $T$ is the wave transmission coefficient through the evanescent zone. The value of $T$ can be calculated via
\begin{equation}
\label{eqn:T}
T  = \exp \bigg[ - \int^{r_2}_{r_1} dr \sqrt{ - \frac{ \big( L_\ell^2 - \omega^2 \big) \big(N^2 - \omega^2 \big) }{v_s^2 \omega^2} } \bigg] \, .
\end{equation}
Here, $r_1$ and $r_2$ are the lower and upper boundaries of the evanescent zone, $L_\ell^2 = l(l+1)v_s^2/r^2$ is the Lamb frequency squared, $N$ is the Brunt-Vaisala frequency, $\omega$ is the angular wave frequency, and $v_s$ is the sound speed. We calculate $\Delta \nu$ and the frequency of maximum power $\nu_{\rm max}$ using  scaling relations.